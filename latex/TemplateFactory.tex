\chapter{TemplateFactory}
Creates javascript function templates from a given ROOT class using TClassRef. Methods and static members are set during creation through use of ROOT reflections and the proxy factories. 
The created templates are kept in a cache to avoid unnecessary creation of already existing templates
\section{createTemplate}
\begin{longtable}{p{3cm} @{\hskip 1cm} p{12cm}}
 \hline
\textit{Name} & \texttt{TemplateFactory::createTemplate(clazz: TClassRef)}\\
\hline
 \textit{Visibility} & public\\
\hline
\textit{Parameters} & \textit{clazz: TClassRef} the class for which a template is to be created \\
\hline
\textit{Return value} & \textbf{ Local<FunctionTemplate>} the created template\\
  \hline
 \textit{Behavior} & Creates such a template. The following sequence diagram illustrates this process \\
\hline
\end{longtable} \pagebreak
 
