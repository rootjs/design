\chapter{FunctionProxy}
	Acts as a proxy for a ROOT callable (i.e. function or class method). It provides methods to execute such a callable and validate its arguments. It also maintains a map of \texttt{TFunction} - \texttt{CallFunc} entries to cache already used functions.

\section{getCallFunc}
\begin{longtable}{p{3cm} @{\hskip 1cm} p{12cm}}
	\hline

	\textit{Name} & \texttt{FunctionProxy::getCallFunc(method: TFunction*)}\\
	\hline

	\textit{Visibility} & public\\
	\hline

	\textit{Parameters} &  \textit{method: TFunction*}: pointer to the ROOT function for which a proxy 
							is to be created\\
	\hline

	\textit{Return value} & \textbf{CallFunc*} a pointer to the CallFunc object provied by kling\\
	\hline

	\textit{behavior} & gets a pointer to a CallFunc object, which encapsulates the provided TFunction 
			in storage (CallFunc is made available by cling)  to which is used during this class' instanciation\\
	\hline

\end{longtable} \pagebreak

\section{getMethodsFromName}
\begin{longtable}{p{3cm} @{\hskip 1cm} p{12cm}}
	\hline

	\textit{Name} & \texttt{FunctionProxy::getMethodsFromName(scope: TClassRef, name: string)}\\
	\hline

	\textit{Visibility} & public\\
	\hline

	\textit{Parameters} & \textit{scope: TClassRef} a reference to the class which is checked for methods with the specified name\\
		& \textit{name: string} the name of the overloaded methods which shall be returned\\
	\hline

	\textit{Return value} & \textbf{vector<TFunction*>} all methods that match the specified name\\
	\hline

	\textit{Behavior} & Gets a reference to a class and a method name string. It returnes all methods of the class with the specified name. This is needed since JavaScript does not support method overloading.\\
	\hline

\end{longtable} \pagebreak

\section{FunctionProxy}
\begin{longtable}{p{3cm} @{\hskip 1cm} p{12cm}}
	\hline

	\textit{Name} & \texttt{FunctionProxy::FunctionProxy(address: void*, function: TFunction, scope: TClassRef)}\\
	\hline

	\textit{Visibility} & public\\
	\hline

	\textit{Parameters} & \textit{address: void*} the memory address of the proxied function\\
		& \textit{function: TFunction} the function's reflection object\\
		& \textit{scope: TClassRef} the class that the function belogs to\\
	\hline

	\textit{Return value} & \textbf{ <<constructor>>} the created \texttt{FunctionProxy}\\
	\hline

	\textit{behavior} & Creates the \texttt{FunctionProxy}.\\
	\hline

\end{longtable} \pagebreak

\section{getType}
\begin{longtable}{p{3cm} @{\hskip 1cm} p{12cm}}
	\hline

	\textit{Name} & \texttt{FunctionProxy::getType()}\\
	\hline

	\textit{Visibility} & public\\
	\hline

	\textit{Parameters} & \textit{none}\\
	\hline

	\textit{Return value} & \textbf{ TFunction} the TFunction object in the proxy \\
	\hline

	\textit{Behavior} & returns the TFunction object this proxy wraps. 
			It contains the meta data of its corresponding function \\
	\hline

\end{longtable} \pagebreak

\section{validateArgs}
\begin{longtable}{p{3cm} @{\hskip 1cm} p{12cm}}
	\hline

	\textit{Name} & \texttt{FunctionProxy::validateArgs(args: FunctionCallbackInfo)}\\
	\hline

	\textit{Visibility} & public\\
	\hline

	\textit{Parameters} & \textit{args: FunctionCallbackInfo} information about the context of the call, including the  number and values of arguments \\
	\hline

	\textit{Return value} & \textbf{ ObjectProxy[]} array of the arguments as proxies\\
	\hline

	\textit{Behavior} & checks whether the functiion is being called with the proper arguments and wraps them in proxies so they can be used by the call method \\
	\hline

\end{longtable} \pagebreak

\section{call}
\begin{longtable}{p{3cm} @{\hskip 1cm} p{12cm}}
	\hline

	\textit{Name} & \texttt{FunctionProxy::call(args: ObjectProxy[])}\\
	\hline

	\textit{Visibility} & public\\
	\hline

	\textit{Parameters} & \textit{args: ObjectProxy[]} proxies containing arguments for the  method\\
	\hline

	\textit{Return value} & \textbf{ ObjectProxy} proxies containing the values returned by the called method\\
	\hline

	\textit{behavior} & calls the actual method in storage using cling.
			The argument object proxies' contents are read and given to the called method  \\
	\hline

\end{longtable} \pagebreak
