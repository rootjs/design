\chapter{FunctionProxy}
describe class FunctionProxy here
\section{getCallFunc}
\begin{longtable}{p{3cm} @{\hskip 1cm} p{12cm}}
 \hline
\textit{Name} & \texttt{FunctionProxy::getCallFunc(method: TFunction*)}\\
\hline
 \textit{Visibility} & public\\
\hline
\textit{Parameters} &  \textit{method: TFunction*}: pointer to the ROOT function for which a proxy 
						is to be created\\
\hline
\textit{Return value} & \textbf{ CallFunc*} a pointer to the CallFunc object provied by kling\\
  \hline
 \textit{behavior} & gets a pointer to a CallFunc object, which encapsulates the provided TFunction 
		in storage (CallFunc is made available by cling)  to which is used during this class' instanciation\\
\hline
\end{longtable} \pagebreak
 \section{getMethodsFromName}
\begin{longtable}{p{3cm} @{\hskip 1cm} p{12cm}}
 \hline
\textit{Name} & \texttt{FunctionProxy::getMethodsFromName(scope: TClassRef, name: string)}\\
\hline
 \textit{Visibility} & public\\
\hline
\textit{Parameters} & \textit{scope: TClassRef, name: string}\\
\hline
\textit{Return value} & \textbf{ vector<TFunction*>} describe return value\\
  \hline
 \textit{behavior} & describe beahviour \\
\hline
\end{longtable} \pagebreak
 \section{FunctionProxy}
\begin{longtable}{p{3cm} @{\hskip 1cm} p{12cm}}
 \hline
\textit{Name} & \texttt{FunctionProxy::FunctionProxy(address: void*, function: TFunction, scope: TClassRef)}\\
\hline
 \textit{Visibility} & public\\
\hline
\textit{Parameters} & \textit{address: void*, function: TFunction, scope: TClassRef}\\
\hline
\textit{Return value} & \textbf{ <<constructor>>} describe return value\\
  \hline
 \textit{behavior} & describe beahviour \\
\hline
\end{longtable} \pagebreak
 \section{getType}
\begin{longtable}{p{3cm} @{\hskip 1cm} p{12cm}}
 \hline
\textit{Name} & \texttt{FunctionProxy::getType()}\\
\hline
 \textit{Visibility} & public\\
\hline
\textit{Parameters} & \textit{none}\\
\hline
\textit{Return value} & \textbf{ TFunction} describe return value\\
  \hline
 \textit{behavior} & describe beahviour \\
\hline
\end{longtable} \pagebreak
 \section{validateArgs}
\begin{longtable}{p{3cm} @{\hskip 1cm} p{12cm}}
 \hline
\textit{Name} & \texttt{FunctionProxy::validateArgs(args: FunctionCallbackInfo)}\\
\hline
 \textit{Visibility} & public\\
\hline
\textit{Parameters} & \textit{args: FunctionCallbackInfo}\\
\hline
\textit{Return value} & \textbf{ ObjectProxy[]} describe return value\\
  \hline
 \textit{behavior} & describe beahviour \\
\hline
\end{longtable} \pagebreak
 \section{call}
\begin{longtable}{p{3cm} @{\hskip 1cm} p{12cm}}
 \hline
\textit{Name} & \texttt{FunctionProxy::call(args: ObjectProxy[])}\\
\hline
 \textit{Visibility} & public\\
\hline
\textit{Parameters} & \textit{args: ObjectProxy[]}\\
\hline
\textit{Return value} & \textbf{ ObjectProxy} describe return value\\
  \hline
 \textit{behavior} & describe beahviour \\
\hline
\end{longtable} \pagebreak
 
