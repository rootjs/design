\chapter{CallbackHandler}
The CallbackHandler class gets invoked whenever an encapsulated ROOT function or object is accessed.
The callback functions follow one general pattern, when called from a Node.JS program \textit{CallbackInfo} is provided.
In the initialization phase we can save \textit{InternalField}s which are belonging to these \textit{CallbackInfo}s.
The internal fields are therefore filled with information about the associated ROOT functionality.
The callback function uses this information to determine what to do exactly.

An inheritor of Proxy will be used to access the data or call the function / constructor and generate a Node.JS representation of the value to be returned.
\section{ctorCallback}
\begin{longtable}{p{3cm} @{\hskip 1cm} p{12cm}}
 \hline
\textit{Name} & \texttt{CallbackHandler::ctorCallback(args: FunctionCallbackInfo<Value>)}\\
\hline
 \textit{Visibility} & public\\
\hline
\textit{Parameters} & \textit{args: FunctionCallbackInfo<Value>} information about the context\\
\hline
\textit{Return value} & \textbf{none}\\
  \hline
 \textit{Behavior} &
 \textit{Whenever the last parameter passed via JavaScript is a JavaScript function, it will be handled as a callback. The following will be done in a new thread, the result is then being passed as a parameter when the callback is being called.}\\

 & This method gets invoked whenever a constructor function of an encapsulated ROOT class is being called.
 This method should decide which constructor should be invoked, by checking constructor overloads.

 This constructor needs to be called and the resulting object needs to be forwarded to the ProxyObjectFactory in order to Proxy the results.\\
\hline
\end{longtable}
\section{memberGetterCallback}
\begin{longtable}{p{3cm} @{\hskip 1cm} p{12cm}}
 \hline
\textit{Name} & \texttt{CallbackHandler::memberGetterCallback(property: Local<String>, info: PropertyCallbackInfo<Value>)}\\
\hline
 \textit{Visibility} & public\\
\hline
\textit{Parameters} & \textit{property: Local<String>, info: PropertyCallbackInfo<Value>}\\
\hline
\textit{Return value} & \textbf{none}\\
  \hline
 \textit{Behavior} & Gets invoked whenever an encapsulated (class) member was requested. The function will not be mapped to a JavaScript function, but to a getter that is being invoked whenever a variable is requested.

 Therefore we do not need to provide the ability to use callbacks here, they could not be passed. In addition to that, it should not take long to read a variable. \\
\hline
\end{longtable}
 \section{memberSetterCallback}
\begin{longtable}{p{3cm} @{\hskip 1cm} p{12cm}}
 \hline
\textit{Name} & \texttt{CallbackHandler::memberSetterCallback(property: Local<String>, value: Local<Value>, info: PropertyCallbackInfo<Value>)}\\
\hline
 \textit{Visibility} & public\\
\hline
\textit{Parameters} & \textit{property: Local<String>, value: Local<Value>, info: PropertyCallbackInfo<Value>}\\
\hline
\textit{Return value} & \textbf{none}\\
  \hline
 \textit{Behavior} & Gets invoked whenever an encapsulated (class) member is attempted to be set. The function will not be mapped to a JavaScript function but to a setter that is being invoked whenever a variable is saved using the \"=\" operator.

 Therefore we do not need to provide the ability to use callbacks here, they could not be passed. In addition to that, it should not take long to write to a variable. \\
\hline
\end{longtable}
 \section{memberFunctionCallback}
\begin{longtable}{p{3cm} @{\hskip 1cm} p{12cm}}
 \hline
\textit{Name} & \texttt{CallbackHandler::memberFunctionCallback(args: FunctionCallbackInfo<Value>)}\\
\hline
 \textit{Visibility} & public\\
\hline
\textit{Parameters} & \textit{args: FunctionCallbackInfo<Value>}\\
\hline
\textit{Return value} & \textbf{none}\\
  \hline
 \textit{Behavior} & \textit{Whenever the last parameter passed via JavaScript is a JavaScript function, it will be handled as a callback. The following will be done in a new thread, the result is then being passed as a parameter when the callback is being called.}\\

 & This method gets invoked whenever a method is called. First a method with the correct signature is selected.
 The selected method will be called and the result will be send trough the ProxyObjectFactory. \\
\hline
\end{longtable} \pagebreak
 \section{staticGetterCallback}
\begin{longtable}{p{3cm} @{\hskip 1cm} p{12cm}}
 \hline
\textit{Name} & \texttt{CallbackHandler::staticGetterCallback(property: Local<String>, info: PropertyCallbackInfo<Value>)}\\
\hline
 \textit{Visibility} & public\\
\hline
\textit{Parameters} & \textit{property: Local<String>, info: PropertyCallbackInfo<Value>}\\
\hline
\textit{Return value} & \textbf{none}\\
  \hline
 \textit{Behavior} &  Gets invoked whenever an encapsulated static property was requested. The function will not be mapped to a JavaScript function, but to a getter that is being invoked whenever a variable is requested.

 Therefore we do not need to provide the ability to use callbacks here, they could not be passed. In addition to that, it should not take long to read a variable. \\
\hline
\end{longtable}
 \section{staticSetterCallback}
\begin{longtable}{p{3cm} @{\hskip 1cm} p{12cm}}
 \hline
\textit{Name} & \texttt{CallbackHandler::staticSetterCallback(property: Local<String>, value: Local<Value>, info: PropertyCallbackInfo<Value>)}\\
\hline
 \textit{Visibility} & public\\
\hline
\textit{Parameters} & \textit{property: Local<String>, value: Local<Value>, info: PropertyCallbackInfo<Value>}\\
\hline
\textit{Return value} & \textbf{none}\\
  \hline
 \textit{Behavior} & Gets invoked whenever an encapsulated static property is attempted to be set. The function will not be mapped to a JavaScript function but to a setter that is being invoked whenever a variable is saved using the \"=\" operator.

 Therefore we do not need to provide the ability to use callbacks here, they could not be passed. In addition to that, it should not take long to write to a variable. \\
\hline
\end{longtable} \pagebreak
 \section{staticFunctionCallback}
\begin{longtable}{p{3cm} @{\hskip 1cm} p{12cm}}
 \hline
\textit{Name} & \texttt{CallbackHandler::staticFunctionCallback(args: FunctionCallbackInfo<Value>)}\\
\hline
 \textit{Visibility} & public\\
\hline
\textit{Parameters} & \textit{args: FunctionCallbackInfo<Value>}\\
\hline
\textit{Return value} & \textbf{none}\\
  \hline
 \textit{Behavior} & \textit{Whenever the last parameter passed via JavaScript is a JavaScript function, it will be handled as a callback. The following will be done in a new thread, the result is then being passed as a parameter when the callback is being called.}\\

 & This method gets invoked whenever a static method is called. First a method with the correct signature is selected.
 The selected method will be called and the result will be send trough the ProxyObjectFactory. \\
\hline
\end{longtable} \pagebreak
