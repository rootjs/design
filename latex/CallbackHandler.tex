\chapter{CallbackHandler}
The CallbackHandler class gets invoked whenever an encapsulated ROOT function or object is accessed.
The callback functions follow one general pattern, when called from a nodeJS program \textit{CallbackInfo} is provided.
In the initialization phase we can save \textit{InternalField}s which are belonging to these \textit{CallbackInfo}s.
The internal fields are therefore filled with information about the associated ROOT functionality.
The callback function uses this information to determine what to do exactly.

An inheritant of Proxy will be used to access the data or call the function / constructor and generate a nodeJS representation of the value to be returned.
\section{ctorCallback}
\begin{longtable}{p{3cm} @{\hskip 1cm} p{12cm}}
 \hline
\textit{Name} & \texttt{CallbackHandler::ctorCallback(args: FunctionCallbackInfo<Value>)}\\
\hline
 \textit{Visibility} & public\\
\hline
\textit{Parameters} & \textit{args: FunctionCallbackInfo<Value>} information about the context\\
\hline
\textit{Return value} & \textbf{none}\\
  \hline
 \textit{Behavior} & Gets invoked whenever a non static constructor function of an encapsulated ROOT class was called.\\
\hline
\end{longtable}
 \section{staticCtorCallback}
\begin{longtable}{p{3cm} @{\hskip 1cm} p{12cm}}
 \hline
\textit{Name} & \texttt{CallbackHandler::staticCtorCallback(args: FunctionCallbackInfo<Value>)}\\
\hline
 \textit{Visibility} & public\\
\hline
\textit{Parameters} & \textit{args: FunctionCallbackInfo<Value>} information about the context\\
\hline
\textit{Return value} & \textbf{none}\\
  \hline
 \textit{Behavior} & Gets invoked whenever a static constructor of an encapsulated ROOT class was called.\\
\hline
\end{longtable}\pagebreak
 \section{memberGetterCallback}
\begin{longtable}{p{3cm} @{\hskip 1cm} p{12cm}}
 \hline
\textit{Name} & \texttt{CallbackHandler::memberGetterCallback(property: Local<String>, info: PropertyCallbackInfo<Value>)}\\
\hline
 \textit{Visibility} & public\\
\hline
\textit{Parameters} & \textit{property: Local<String>, info: PropertyCallbackInfo<Value>}\\
\hline
\textit{Return value} & \textbf{none}\\
  \hline
 \textit{Behavior} & Gets invoked whenever an encapsulated (class) member was requested.\\
\hline
\end{longtable}
 \section{memberSetterCallback}
\begin{longtable}{p{3cm} @{\hskip 1cm} p{12cm}}
 \hline
\textit{Name} & \texttt{CallbackHandler::memberSetterCallback(property: Local<String>, value: Local<Value>, info: PropertyCallbackInfo<Value>)}\\
\hline
 \textit{Visibility} & public\\
\hline
\textit{Parameters} & \textit{property: Local<String>, value: Local<Value>, info: PropertyCallbackInfo<Value>}\\
\hline
\textit{Return value} & \textbf{none}\\
  \hline
 \textit{Behavior} & Gets invoked whenever an encapsulated (class) member is attempted to be set.\\
\hline
\end{longtable}
 \section{memberFunctionCallback}
\begin{longtable}{p{3cm} @{\hskip 1cm} p{12cm}}
 \hline
\textit{Name} & \texttt{CallbackHandler::memberFunctionCallback(args: FunctionCallbackInfo<Value>)}\\
\hline
 \textit{Visibility} & public\\
\hline
\textit{Parameters} & \textit{args: FunctionCallbackInfo<Value>}\\
\hline
\textit{Return value} & \textbf{none}\\
  \hline
 \textit{Behavior} & Gets invoked whenever an non-static (class) function was called.\\
\hline
\end{longtable} \pagebreak
 \section{staticGetterCallback}
\begin{longtable}{p{3cm} @{\hskip 1cm} p{12cm}}
 \hline
\textit{Name} & \texttt{CallbackHandler::staticGetterCallback(property: Local<String>, info: PropertyCallbackInfo<Value>)}\\
\hline
 \textit{Visibility} & public\\
\hline
\textit{Parameters} & \textit{property: Local<String>, info: PropertyCallbackInfo<Value>}\\
\hline
\textit{Return value} & \textbf{none}\\
  \hline
 \textit{Behavior} &  Gets invoked whenever an encapsulated static object was requested.\\
\hline
\end{longtable}
 \section{staticSetterCallback}
\begin{longtable}{p{3cm} @{\hskip 1cm} p{12cm}}
 \hline
\textit{Name} & \texttt{CallbackHandler::staticSetterCallback(property: Local<String>, value: Local<Value>, info: PropertyCallbackInfo<Value>)}\\
\hline
 \textit{Visibility} & public\\
\hline
\textit{Parameters} & \textit{property: Local<String>, value: Local<Value>, info: PropertyCallbackInfo<Value>}\\
\hline
\textit{Return value} & \textbf{none}\\
  \hline
 \textit{Behavior} & Gets invoked whenever an encapsulated static object is attempted to be set.\\
\hline
\end{longtable} 
 \section{staticFunctionCallback}
\begin{longtable}{p{3cm} @{\hskip 1cm} p{12cm}}
 \hline
\textit{Name} & \texttt{CallbackHandler::staticFunctionCallback(args: FunctionCallbackInfo<Value>)}\\
\hline
 \textit{Visibility} & public\\
\hline
\textit{Parameters} & \textit{args: FunctionCallbackInfo<Value>}\\
\hline
\textit{Return value} & \textbf{none}\\
  \hline
 \textit{Behavior} & Gets invoked whenever a static function was called.\\
\hline
\end{longtable} \pagebreak
