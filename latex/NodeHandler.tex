\chapter{NodeHandler}
The \textit{NodeHandler} is the main entry point, when you require RootJS by using
\begin{verbatim}
// JavaScript: Load ROOT bindings in JavaScript
var root = require(rootJS.node);

// C++: Expose the initialize method as the main entry point
NODE_MODULE(rootJS, initialize)
\end{verbatim}
after running the initialize method ROOT is fully initialized and all features are exposed to JavaScript.
\section{initialize}
\begin{longtable}{p{3cm} @{\hskip 1cm} p{12cm}}
 \hline
\textit{Name} & \texttt{NodeHandler::initialize(exports: Local<Object>, module: Local<Object>)}\\
\hline
 \textit{Visibility} & public static\\
\hline
\textit{Parameters} & \textit{exports: Local<Object>, module: Local<Object>} parameters passed by NodeJS\\
\hline
\textit{Return value} & \textit{none} The features will be exported by passing them to the exports parameter \\
  \hline
 \textit{Behavior} & This will create an instance of \textit{NodeApplication} and store it in gApplication, to ensure all ROOT functionallity that relies on gAppllication will work.
 Further this will run \textit{getExports} to retrive the features to be exported to JavaScript which will then be put into the exports object which has been passed to this method \\
\hline
\end{longtable} 
\section{getExports}
\begin{longtable}{p{3cm} @{\hskip 1cm} p{12cm}}
 \hline
\textit{Name} & \texttt{NodeHandler::getExports()}\\
\hline
 \textit{Visibility} & public\\
\hline
\textit{Parameters} & \textit{none}\\
\hline
\textit{Return value} & \textbf{ Local<Object>} features to be exported \\
  \hline
 \textit{Behavior} & This method will run multiple private methods to collect global function, global variables, macros and classes.
 All these items will be stored in a v8 object which will be passed to RootJS via the initialize method. \\
\hline
\end{longtable} \pagebreak
