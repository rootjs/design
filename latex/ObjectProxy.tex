\chapter{ObjectProxy}
The \textit{ObjectProxy} class is used to represent ROOT objects. It differentiates between primitive and non-primitive object types.
\section{ObjectProxy}
\begin{longtable}{p{3cm} @{\hskip 1cm} p{12cm}}
 \hline
\textit{Name} & \texttt{ObjectProxy::ObjectProxy(type: TDataMember, scope: TClassRef)}\\
\hline
 \textit{Visibility} & public\\
\hline
\textit{Parameters} & \textit{type: TDataMember, scope: TClassRef}\\
\hline
\textit{Return value} & \textbf{<<constructor>>} The newly constructed ObjectProxy.\\
  \hline
  \textit{behavior} & Creates a new ObjectProxy with the given type and \textit{TClassRef}.\\
\hline
\end{longtable} \pagebreak
 \section{getType}
\begin{longtable}{p{3cm} @{\hskip 1cm} p{12cm}}
 \hline
\textit{Name} & \texttt{ObjectProxy::getType()}\\
\hline
 \textit{Visibility} & public\\
\hline
\textit{Parameters} & \textit{none}\\
\hline
\textit{Return value} & \textbf{TDataMember} The type of the ObjectProxy.\\
  \hline
 \textit{behavior} & Returns the type of the Object behind the proxy.\\
\hline
\end{longtable} \pagebreak
 \section{set}
\begin{longtable}{p{3cm} @{\hskip 1cm} p{12cm}}
 \hline
\textit{Name} & \texttt{ObjectProxy::set(value: ObjectProxy)}\\
\hline
 \textit{Visibility} & public\\
\hline
\textit{Parameters} & \textit{value: ObjectProxy}\\
\hline
\textit{Return value} & \textbf{none}\\
  \hline
 \textit{behavior} & Sets the value of the Object behind the proxy.\\
\hline
\end{longtable} \pagebreak
 \section{get}
\begin{longtable}{p{3cm} @{\hskip 1cm} p{12cm}}
 \hline
\textit{Name} & \texttt{ObjectProxy::get()}\\
\hline
 \textit{Visibility} & public\\
\hline
\textit{Parameters} & \textit{none}\\
\hline
\textit{Return value} & \textbf{Local<Value>} The value the object has.\\
  \hline
 \textit{behavior} & Returns the value that was set for the object.\\
\hline
\end{longtable} \pagebreak
 \section{setProxy}
\begin{longtable}{p{3cm} @{\hskip 1cm} p{12cm}}
 \hline
\textit{Name} & \texttt{ObjectProxy::setProxy(proxy: Local<Object>)}\\
\hline
 \textit{Visibility} & public\\
\hline
\textit{Parameters} & \textit{proxy: Local<Object>}\\
\hline
\textit{Return value} & \textbf{none}\\
  \hline
 \textit{behavior} & describe beahviour \\
\hline
\end{longtable} \pagebreak
 \section{getProxy}
\begin{longtable}{p{3cm} @{\hskip 1cm} p{12cm}}
 \hline
\textit{Name} & \texttt{ObjectProxy::getProxy()}\\
\hline
 \textit{Visibility} & public\\
\hline
\textit{Parameters} & \textit{none}\\
\hline
\textit{Return value} & \textbf{Local<Object>} describe return value\\
  \hline
 \textit{behavior} & describe beahviour \\
\hline
\end{longtable} \pagebreak
 \section{isPrimitive}
\begin{longtable}{p{3cm} @{\hskip 1cm} p{12cm}}
 \hline
\textit{Name} & \texttt{ObjectProxy::isPrimitive()}\\
\hline
 \textit{Visibility} & public\\
\hline
\textit{Parameters} & \textit{none}\\
\hline
\textit{Return value} & \textbf{bool} Whether or not the represented object is of a primitive type or not.\\
  \hline
  \textit{behavior} & Returns \textit{true} if the represented object's type is primitive, \textit{false} if not.\\
\hline
\end{longtable} \pagebreak
