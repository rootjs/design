\chapter{TemplateCache.tex}
describe class TemplateCache.tex here
\section{contains}
\begin{longtable}{p{3cm} @{\hskip 1cm} p{12cm}}
 \hline
\textit{Name} & \texttt{TemplateCache.tex::contains(type: TClassRef)}\\
\hline
 \textit{Visibility} & public\\
\hline
\textit{Parameters} & \textit{type: TClassRef}\\
\hline
\textit{Return value} & \textbf{ bool} describe return value\\
  \hline
 \textit{behavior} & describe beahviour \\
\hline
\end{longtable} \pagebreak
 \section{get}
\begin{longtable}{p{3cm} @{\hskip 1cm} p{12cm}}
 \hline
\textit{Name} & \texttt{TemplateCache.tex::get(type: TClassRef)}\\
\hline
 \textit{Visibility} & public\\
\hline
\textit{Parameters} & \textit{type: TClassRef}\\
\hline
\textit{Return value} & \textbf{ Local<FunctionTemplate>} describe return value\\
  \hline
 \textit{behavior} & describe beahviour \\
\hline
\end{longtable} \pagebreak
 \section{store}
\begin{longtable}{p{3cm} @{\hskip 1cm} p{12cm}}
 \hline
\textit{Name} & \texttt{TemplateCache.tex::store(type: TClassRef, tpl: Local<FunctionTemplate>)}\\
\hline
 \textit{Visibility} & public\\
\hline
\textit{Parameters} & \textit{type: TClassRef, tpl: Local<FunctionTemplate>}\\
\hline
\textit{Return value} & \textbf{none}\\
  \hline
 \textit{behavior} & describe beahviour \\
\hline
\end{longtable} \pagebreak
 