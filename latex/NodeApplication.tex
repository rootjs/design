\chapter{NodeApplication}
ROOT uses TApplication to interface with the windowing system and event handlers.
An insatnce of TApplication is usually stored in the global \textit{gApplication} variable.

The main problem with using TApplication directly would be, that we could not hook into the \textit{InitializeGraphics} method.
When having a graphical user interface we need to do a UI update frequently:
\begin{verbatim}
gSystem->ProcessEvents();
\end{verbatim}
To avoid having a lot of \textit{ProcessEvents} calls, we wait until \textit{InitializeGraphics} has been called at least once.

Further NodeApplication is being used to set the application's name and initialize a custom message callback which can be used to retrive messages in JavaScript.

\section{NodeApplication}
\begin{longtable}{p{3cm} @{\hskip 1cm} p{12cm}}
 \hline
\textit{Name} & \texttt{NodeApplication::NodeApplication(acn: char*, argc: int*, argv: char**)}\\
\hline
 \textit{Visibility} & public\\
\hline
\textit{Parameters} & \textit{acn: char*, argc: int*, argv: char**}\\
\hline
\textit{Return value} & \textbf{<<constructor>>} describe return value\\
  \hline
 \textit{Behavior} & Set's the application name and constructs a custom message handler \\
\hline
\end{longtable} \pagebreak
