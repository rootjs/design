testpl.puml
\chapter{NodeApplication}
describe class NodeApplication here
\section{ctorCallback}
\begin{longtable}{p{3cm} @{\hskip 1cm} p{12cm}}
 \hline
\textit{Name} & \texttt\{NodeApplication::ctorCallback(args: FunctionCallbackInfo<Value>)}\\
\hline
 \textit{Visibility} & \\
\hline
\textit{Parameters} & \textit{args: FunctionCallbackInfo<Value>}\\
\hline
\textit{Return value} & \textbf{} describe return value\\
  \hline
  \textit{behavior} & describe beahviour \\
  \hline
\end{longtable}
 \pagebreak\n\section{staticCtorCallback}
\begin{longtable}{p{3cm} @{\hskip 1cm} p{12cm}}
 \hline
\textit{Name} & \texttt\{NodeApplication::staticCtorCallback(args: FunctionCallbackInfo<Value>)}\\
\hline
 \textit{Visibility} & \\
\hline
\textit{Parameters} & \textit{args: FunctionCallbackInfo<Value>}\\
\hline
\textit{Return value} & \textbf{} describe return value\\
  \hline
  \textit{behavior} & describe beahviour \\
  \hline
\end{longtable}
 \pagebreak\n\section{memberGetterCallback}
\begin{longtable}{p{3cm} @{\hskip 1cm} p{12cm}}
 \hline
\textit{Name} & \texttt\{NodeApplication::memberGetterCallback(property: Local<String>, info: PropertyCallbackInfo<Value>)}\\
\hline
 \textit{Visibility} & \\
\hline
\textit{Parameters} & \textit{property: Local<String>, info: PropertyCallbackInfo<Value>}\\
\hline
\textit{Return value} & \textbf{} describe return value\\
  \hline
  \textit{behavior} & describe beahviour \\
  \hline
\end{longtable}
 \pagebreak\n\section{memberSetterCallback}
\begin{longtable}{p{3cm} @{\hskip 1cm} p{12cm}}
 \hline
\textit{Name} & \texttt\{NodeApplication::memberSetterCallback(property: Local<String>, value: Local<Value>, info: PropertyCallbackInfo<Value>)}\\
\hline
 \textit{Visibility} & \\
\hline
\textit{Parameters} & \textit{property: Local<String>, value: Local<Value>, info: PropertyCallbackInfo<Value>}\\
\hline
\textit{Return value} & \textbf{} describe return value\\
  \hline
  \textit{behavior} & describe beahviour \\
  \hline
\end{longtable}
 \pagebreak\n\section{memberFunctionCallback}
\begin{longtable}{p{3cm} @{\hskip 1cm} p{12cm}}
 \hline
\textit{Name} & \texttt\{NodeApplication::memberFunctionCallback(args: FunctionCallbackInfo<Value>)}\\
\hline
 \textit{Visibility} & \\
\hline
\textit{Parameters} & \textit{args: FunctionCallbackInfo<Value>}\\
\hline
\textit{Return value} & \textbf{} describe return value\\
  \hline
  \textit{behavior} & describe beahviour \\
  \hline
\end{longtable}
 \pagebreak\n\section{staticGetterCallback}
\begin{longtable}{p{3cm} @{\hskip 1cm} p{12cm}}
 \hline
\textit{Name} & \texttt\{NodeApplication::staticGetterCallback(property: Local<String>, info: PropertyCallbackInfo<Value>)}\\
\hline
 \textit{Visibility} & \\
\hline
\textit{Parameters} & \textit{property: Local<String>, info: PropertyCallbackInfo<Value>}\\
\hline
\textit{Return value} & \textbf{} describe return value\\
  \hline
  \textit{behavior} & describe beahviour \\
  \hline
\end{longtable}
 \pagebreak\n\section{staticSetterCallback}
\begin{longtable}{p{3cm} @{\hskip 1cm} p{12cm}}
 \hline
\textit{Name} & \texttt\{NodeApplication::staticSetterCallback(property: Local<String>, value: Local<Value>, info: PropertyCallbackInfo<Value>)}\\
\hline
 \textit{Visibility} & \\
\hline
\textit{Parameters} & \textit{property: Local<String>, value: Local<Value>, info: PropertyCallbackInfo<Value>}\\
\hline
\textit{Return value} & \textbf{} describe return value\\
  \hline
  \textit{behavior} & describe beahviour \\
  \hline
\end{longtable}
 \pagebreak\n\section{staticFunctionCallback}
\begin{longtable}{p{3cm} @{\hskip 1cm} p{12cm}}
 \hline
\textit{Name} & \texttt\{NodeApplication::staticFunctionCallback(args: FunctionCallbackInfo<Value>)}\\
\hline
 \textit{Visibility} & \\
\hline
\textit{Parameters} & \textit{args: FunctionCallbackInfo<Value>}\\
\hline
\textit{Return value} & \textbf{} describe return value\\
  \hline
  \textit{behavior} & describe beahviour \\
  \hline
\end{longtable}
 \pagebreak\n\section{Instance}
\begin{longtable}{p{3cm} @{\hskip 1cm} p{12cm}}
 \hline
\textit{Name} & \texttt\{NodeApplication::Instance()}\\
\hline
 \textit{Visibility} & \\
\hline
\textit{Parameters} & \textit{none}\\
\hline
\textit{Return value} & \textbf{ NodeApplicationnone}\section{init}
\begin{longtable}{p{3cm} @{\hskip 1cm} p{12cm}}
 \hline
\textit{Name} & \texttt\{NodeApplication::init(exports: Local<Object>, module: Local<Object>)}\\
\hline
 \textit{Visibility} & \\
\hline
\textit{Parameters} & \textit{exports: Local<Object>, module: Local<Object>}\\
\hline
\textit{Return value} & \textbf{} describe return value\\
  \hline
  \textit{behavior} & describe beahviour \\
  \hline
\end{longtable}
 \pagebreak\n\chapter{TemplateFactory}
describe class TemplateFactory here
\section{createTemplate}
\begin{longtable}{p{3cm} @{\hskip 1cm} p{12cm}}
 \hline
\textit{Name} & \texttt\{TemplateFactory::createTemplate(clazz: TClassRef)}\\
\hline
 \textit{Visibility} & \\
\hline
\textit{Parameters} & \textit{clazz: TClassRef}\\
\hline
\textit{Return value} & \textbf{ Local<FunctionTemplate>} describe return value\\
  \hline
  \textit{behavior} & describe beahviour \\
  \hline
\end{longtable}
 \pagebreak\n\chapter{TemplateCache}
describe class TemplateCache here
\section{contains}
\begin{longtable}{p{3cm} @{\hskip 1cm} p{12cm}}
 \hline
\textit{Name} & \texttt\{TemplateCache::contains(type: TClassRef)}\\
\hline
 \textit{Visibility} & \\
\hline
\textit{Parameters} & \textit{type: TClassRef}\\
\hline
\textit{Return value} & \textbf{ bool} describe return value\\
  \hline
  \textit{behavior} & describe beahviour \\
  \hline
\end{longtable}
 \pagebreak\n\section{get}
\begin{longtable}{p{3cm} @{\hskip 1cm} p{12cm}}
 \hline
\textit{Name} & \texttt\{TemplateCache::get(type: TClassRef)}\\
\hline
 \textit{Visibility} & \\
\hline
\textit{Parameters} & \textit{type: TClassRef}\\
\hline
\textit{Return value} & \textbf{ Local<FunctionTemplate>} describe return value\\
  \hline
  \textit{behavior} & describe beahviour \\
  \hline
\end{longtable}
 \pagebreak\n\section{store}
\begin{longtable}{p{3cm} @{\hskip 1cm} p{12cm}}
 \hline
\textit{Name} & \texttt\{TemplateCache::store(type: TClassRef, tpl: Local<FunctionTemplate>)}\\
\hline
 \textit{Visibility} & \\
\hline
\textit{Parameters} & \textit{type: TClassRef, tpl: Local<FunctionTemplate>}\\
\hline
\textit{Return value} & \textbf{} describe return value\\
  \hline
  \textit{behavior} & describe beahviour \\
  \hline
\end{longtable}
 \pagebreak\n\chapter{ProxyFunctionFactory}
describe class ProxyFunctionFactory here
\section{createProxyFunction}
\begin{longtable}{p{3cm} @{\hskip 1cm} p{12cm}}
 \hline
\textit{Name} & \texttt\{ProxyFunctionFactory::createProxyFunction(info: TMethod)}\\
\hline
 \textit{Visibility} & \\
\hline
\textit{Parameters} & \textit{info: TMethod}\\
\hline
\textit{Return value} & \textbf{ ProxyFunciton} describe return value\\
  \hline
  \textit{behavior} & describe beahviour \\
  \hline
\end{longtable}
 \pagebreak\n\section{fromArgs}
\begin{longtable}{p{3cm} @{\hskip 1cm} p{12cm}}
 \hline
\textit{Name} & \texttt\{ProxyFunctionFactory::fromArgs(name: string, clazz: TClassRef, args: FunctionCallbackInfo)}\\
\hline
 \textit{Visibility} & \\
\hline
\textit{Parameters} & \textit{name: string, clazz: TClassRef, args: FunctionCallbackInfo}\\
\hline
\textit{Return value} & \textbf{ ProxyFunction} describe return value\\
  \hline
  \textit{behavior} & describe beahviour \\
  \hline
\end{longtable}
 \pagebreak\n\chapter{ProxyFunction}
describe class ProxyFunction here
\section{ProxyFunction}
\begin{longtable}{p{3cm} @{\hskip 1cm} p{12cm}}
 \hline
\textit{Name} & \texttt\{ProxyFunction::ProxyFunction(address: void*, info: TFunction)}\\
\hline
 \textit{Visibility} & \\
\hline
\textit{Parameters} & \textit{address: void*, info: TFunction}\\
\hline
\textit{Return value} & \textbf{ <<constructor>>} describe return value\\
  \hline
  \textit{behavior} & describe beahviour \\
  \hline
\end{longtable}
 \pagebreak\n\section{convertArgs}
\begin{longtable}{p{3cm} @{\hskip 1cm} p{12cm}}
 \hline
\textit{Name} & \texttt\{ProxyFunction::convertArgs(args: FunctionCallbackInfo)}\\
\hline
 \textit{Visibility} & \\
\hline
\textit{Parameters} & \textit{args: FunctionCallbackInfo}\\
\hline
\textit{Return value} & \textbf{ ProxyObject[]} describe return value\\
  \hline
  \textit{behavior} & describe beahviour \\
  \hline
\end{longtable}
 \pagebreak\n\section{call}
\begin{longtable}{p{3cm} @{\hskip 1cm} p{12cm}}
 \hline
\textit{Name} & \texttt\{ProxyFunction::call(args: ProxyObject[])}\\
\hline
 \textit{Visibility} & \\
\hline
\textit{Parameters} & \textit{args: ProxyObject[]}\\
\hline
\textit{Return value} & \textbf{ ProxyObject} describe return value\\
  \hline
  \textit{behavior} & describe beahviour \\
  \hline
\end{longtable}
 \pagebreak\n\section{isTemplateFunction}
\begin{longtable}{p{3cm} @{\hskip 1cm} p{12cm}}
 \hline
\textit{Name} & \texttt\{ProxyFunction::isTemplateFunction()}\\
\hline
 \textit{Visibility} & \\
\hline
\textit{Parameters} & \textit{none}\\
\hline
\textit{Return value} & \textbf{ boolnone}\chapter{ProxyObjectFactory}
describe class ProxyObjectFactory here
\section{createProxyObject}
\begin{longtable}{p{3cm} @{\hskip 1cm} p{12cm}}
 \hline
\textit{Name} & \texttt\{ProxyObjectFactory::createProxyObject(type: TDataMember, holder: ProxyObject)}\\
\hline
 \textit{Visibility} & \\
\hline
\textit{Parameters} & \textit{type: TDataMember, holder: ProxyObject}\\
\hline
\textit{Return value} & \textbf{ ProxyObject} describe return value\\
  \hline
  \textit{behavior} & describe beahviour \\
  \hline
\end{longtable}
 \pagebreak\n\chapter{ProxyObject}
describe class ProxyObject here
\section{ProxyObject}
\begin{longtable}{p{3cm} @{\hskip 1cm} p{12cm}}
 \hline
\textit{Name} & \texttt\{ProxyObject::ProxyObject(address: void*, type: TDataMember)}\\
\hline
 \textit{Visibility} & \\
\hline
\textit{Parameters} & \textit{address: void*, type: TDataMember}\\
\hline
\textit{Return value} & \textbf{ <<constructor>>} describe return value\\
  \hline
  \textit{behavior} & describe beahviour \\
  \hline
\end{longtable}
 \pagebreak\n\section{getAddress}
\begin{longtable}{p{3cm} @{\hskip 1cm} p{12cm}}
 \hline
\textit{Name} & \texttt\{ProxyObject::getAddress()}\\
\hline
 \textit{Visibility} & \\
\hline
\textit{Parameters} & \textit{none}\\
\hline
\textit{Return value} & \textbf{ void*none}\section{getType}
\begin{longtable}{p{3cm} @{\hskip 1cm} p{12cm}}
 \hline
\textit{Name} & \texttt\{ProxyObject::getType()}\\
\hline
 \textit{Visibility} & \\
\hline
\textit{Parameters} & \textit{none}\\
\hline
\textit{Return value} & \textbf{ TDataMembernone}\section{set}
\begin{longtable}{p{3cm} @{\hskip 1cm} p{12cm}}
 \hline
\textit{Name} & \texttt\{ProxyObject::set(value: ProxyObject)}\\
\hline
 \textit{Visibility} & \\
\hline
\textit{Parameters} & \textit{value: ProxyObject}\\
\hline
\textit{Return value} & \textbf{} describe return value\\
  \hline
  \textit{behavior} & describe beahviour \\
  \hline
\end{longtable}
 \pagebreak\n\section{get}
\begin{longtable}{p{3cm} @{\hskip 1cm} p{12cm}}
 \hline
\textit{Name} & \texttt\{ProxyObject::get()}\\
\hline
 \textit{Visibility} & \\
\hline
\textit{Parameters} & \textit{none}\\
\hline
\textit{Return value} & \textbf{ Local<Value>none}\section{isPrimitive}
\begin{longtable}{p{3cm} @{\hskip 1cm} p{12cm}}
 \hline
\textit{Name} & \texttt\{ProxyObject::isPrimitive()}\\
\hline
 \textit{Visibility} & \\
\hline
\textit{Parameters} & \textit{none}\\
\hline
\textit{Return value} & \textbf{ boolnone}\chapter{ROOTJS}
describe class ROOTJS here
\section{callFunction}
\begin{longtable}{p{3cm} @{\hskip 1cm} p{12cm}}
 \hline
\textit{Name} & \texttt\{ROOTJS::callFunction(func: TFunction*, args: char*)}\\
\hline
 \textit{Visibility} & \\
\hline
\textit{Parameters} & \textit{func: TFunction*, args: char*}\\
\hline
\textit{Return value} & \textbf{ CallFunc_t*} describe return value\\
  \hline
  \textit{behavior} & describe beahviour \\
  \hline
\end{longtable}
 \pagebreak\n\section{callConstructor}
\begin{longtable}{p{3cm} @{\hskip 1cm} p{12cm}}
 \hline
\textit{Name} & \texttt\{ROOTJS::callConstructor(clazz: TClassRef const&, args: char*)}\\
\hline
 \textit{Visibility} & \\
\hline
\textit{Parameters} & \textit{clazz: TClassRef const&, args: char*}\\
\hline
\textit{Return value} & \textbf{ CallFunc_t*} describe return value\\
  \hline
  \textit{behavior} & describe beahviour \\
  \hline
\end{longtable}
 \pagebreak\n\section{isConstructor}
\begin{longtable}{p{3cm} @{\hskip 1cm} p{12cm}}
 \hline
\textit{Name} & \texttt\{ROOTJS::isConstructor(func: TMethod*)}\\
\hline
 \textit{Visibility} & \\
\hline
\textit{Parameters} & \textit{func: TMethod*}\\
\hline
\textit{Return value} & \textbf{ bool} describe return value\\
  \hline
  \textit{behavior} & describe beahviour \\
  \hline
\end{longtable}
 \pagebreak\n