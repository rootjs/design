\chapter{FunctionProxyFactory}
The \textit{FunctionProxyFactory} 
\section{createFunctionProxy}
\begin{longtable}{p{3cm} @{\hskip 1cm} p{12cm}}
 \hline
\textit{Name} & \texttt{FunctionProxyFactory::createFunctionProxy(function: TFunction, scope: TClassRef)}\\
\hline
 \textit{Visibility} & public\\
\hline
\textit{Parameters} & \textit{function: TFunction} The ROOT function to be proxied. \\ 
& \textit{scope: TClassRef}\\
\hline
\textit{Return value} & \textbf{ProxyFunction} the proxied function\\
  \hline
 \textit{Behavior} & Creates \textit{FunctionProxy} objects with \textit{TFunction} function and \textit{TClassRef} scope.\\
\hline
\end{longtable} \pagebreak
 \section{fromArgs}
\begin{longtable}{p{3cm} @{\hskip 1cm} p{12cm}}
 \hline	
\textit{Name} & \texttt{FunctionProxyFactory::fromArgs(name: string, scope: TClassRef, args: FunctionCallbackInfo)}\\
\hline
 \textit{Visibility} & public\\
\hline
\textit{Parameters} & \textit{name: string, scope: TClassRef, args: FunctionCallbackInfo} The name of the function, its scope and the arguments it takes.\\
\hline
\textit{Return value} & \textbf{FunctionProxy} describe return value\\
  \hline
 \textit{Behavior} & This method is called from the \textit{createFunctionProxy} method to deal with overloaded functions, since JavaScript doesn't support it.\\
\hline
\end{longtable} \pagebreak
